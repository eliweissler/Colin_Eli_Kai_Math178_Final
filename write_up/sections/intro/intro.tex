As of 2019, it was estimated that 5 billion people have mobile devices and over half of these devices are smartphones. The sheer amount of data generated from these devices is enormous; most of them are equipped with accelerometer, gyroscopes, magnetometers, etc. The information from these sensors has enabled  the study of Human Activity Recognition (HAR). HAR has a variety of applications including healthcare, sports, continuous user authentication, and biometric key generation. The HAR data is often non-linear, but the approaches used for analysis do not take this into account.

There have been several papers published which utilize machine learning and deep learning techniques for HAR. \citet{masum2019human} used labeled accelerometer, gyroscope, magnetometer and electrocardiogram data for HAR \cite{masum2019human}. They compared the performance of three different classifiers (Random Forest, Support Vector Machine, Naive Bayes) and three different deep neural network architectures (Multilayer Perceptron, Deep Convolutional Neural Network and Long-Short Term Memory), and found that the deep neural networks performed the best. 

\citet{ferrari2019hand}. compared the performance of traditional machine learning methods (k-NN, SVM) to a residual network (ResNet) They used a variety of datasets, including motion sense, the dataset that we have used for this paper. They used only the accelerometer data, and compared the classifier performance for using the raw data vs hand-crafted features. For the motion sense dataset, the hand-crafted features gave the best results for the classifier, but the ResNet still performed the best. ResNets allow for skipping between layers; these skips contain non-linearities, which are better able to represent the non-linear accelerometer data \cite{he2016deep}.

While both papers proved that deep learning architectures are the superior methods for HAR, neither of their traditional approached accounted for the non-linear aspect of the accelerometer data. We are interested in improving the performance of traditional machine learning methods by accurately accounting for the the non-linearity of the data in a pre-processing step. 
