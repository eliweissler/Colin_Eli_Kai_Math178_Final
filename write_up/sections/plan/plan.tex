I think this is the story we largely are trying to tell: 
\begin{enumerate}
	\item Can we use cell phone data to accurately and consistently distinguish between the activities people are doing? The answer had better be yes because it seems loads of other folks have done so.

		\begin{itemize}
	\item What techniques are used to do this? Luckily for us, we have this wonderful set of \emph{labeled} (and somewhat diverse) data. So obviously we are probably interested in looking at how supervised learning techniques perform. The amazing thing is that scikit-learn's syntax is basically identical once the data gets set up so it really becomes a plug and chug type of thing after that. Some techniques includes:
			\item Support vector machines (SVM)
			\item Some deep/crazy-architectured neural network; since this seems to be all the rage these days and since NN claim to be able to catch non-linearness of problems, it seems like a good thing to compare our results to.
			\item  Decision trees?
			\item SGD classifiers
			\item others?
		\end{itemize}

	\item But, what I think Prof.\ Gu is looking for is that we can use manifold learning with linear techniques for an unsupervised set of data. Can we classify accuately with that? Some techniques we may consider
		\begin{itemize}
			\item k-means
			\item PCA
			\item Gaussian mixture
		\end{itemize}

\end{enumerate}
