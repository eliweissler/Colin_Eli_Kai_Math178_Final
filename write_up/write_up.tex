\documentclass{article}

% if you need to pass options to natbib, use, e.g.:
% \PassOptionsToPackage{numbers, compress}{natbib}
% before loading nips_2016
%
% to avoid loading the natbib package, add option nonatbib:
% \usepackage[nonatbib]{nips_2016}

\usepackage[final]{nips_2016}

% to compile a camera-ready version, add the [final] option, e.g.:
% \usepackage[final]{nips_2016}

\usepackage[utf8]{inputenc} % allow utf-8 input
\usepackage[T1]{fontenc}    % use 8-bit T1 fonts
\usepackage[hidelinks=true]{hyperref}       % hyperlinks
\usepackage{url}            % simple URL typesetting
\usepackage{booktabs}       % professional-quality tables
\usepackage{amsfonts}       % blackboard math symbols
\usepackage{nicefrac}       % compact symbols for 1/2, etc.
\usepackage{microtype}      % microtypography
\usepackage[USenglish]{babel}
\usepackage{physics}
\usepackage{siunitx}
\usepackage[labelfont=bf]{caption}
\usepackage{todonotes}


\title{Human Activity Prediction on Nonlinear Manifolds}

% The \author macro works with any number of authors. There are two
% commands used to separate the names and addresses of multiple
% authors: \And and \AND.
%
% Using \And between authors leaves it to LaTeX to determine where to
% break the lines. Using \AND forces a line break at that point. So,
% if LaTeX puts 3 of 4 authors names on the first line, and the last
% on the second line, try using \AND instead of \And before the third
% author name.

\author{
  Colin M. Adams\thanks{All authors contributed equally. The names are ordered alphabetically.} \\
   Detection \& Estimation\\
   Aret\'e Associates\\
   Northridge, CA 91324 \\
   \texttt{cadams@arete.com}

  %% examples of more authors
  \And
  Kai S. Kaneshina \\
   Detection \& Estimation\\
   Aret\'e Associates\\
   Northridge, CA 91324 \\
   \texttt{kkaneshina@arete.com} 
   %comment below to be less cheeky
  \And
  Eli J. Weissler \\
   Detection \& Estimation\\
   Aret\'e Associates\\
   Northridge, CA 91324 \\
   \texttt{eweissler@arete.com}
   %comment below to be less cheeky
}

\begin{document}
% \nipsfinalcopy is no longer used

\maketitle

\begin{abstract}
  The abstract paragraph should be indented \nicefrac{1}{2}~inch (3~picas) on both the left- and right-hand margins. Use 10~point  type, with a vertical spacing (leading) of 11~points.  The word \textbf{Abstract} must be centered, bold, and in point size 12. Two line spaces precede the abstract. The abstract must be limited to one paragraph.
\end{abstract}

\section{Progress Report}
\textbf{Instructions from the course website say:}\\
{\itshape Submit the progress report detailing your progress towards your goal. Typed (LaTeX) summarizing your literature search, specifying what data sets you are using, and what methods you are applying. The write-up should be 3 to 5 pages for a 1 person group, 6 to 8 pages for a 2 person group and 8 to 10 for a 3 person group. }


\section{Introduction}
  As of 2019, it was estimated that 5 billion people have mobile devices and over half of these devices are smartphones. The sheer amount of data generated from these devices is enormous; most of them are equipped with accelerometer, gyroscopes, magnetometers, etc. The information from these sensors has enabled  the study of Human Activity Recognition (HAR). HAR has a variety of applications including healthcare, sports, continuous user authentication, and biometric key generation. The HAR data is often non-linear, but the approaches used for analysis do not take this into account.

There have been several papers published which utilize machine learning and deep learning techniques for HAR. Previous researchers used labeled accelerometer, gyroscope, magnetometer and electrocardiogram data for HAR \cite{masum2019human}. They compared the performance of three different classifiers (Random Forest, Support Vector Machine, Naive Bayes) and three different deep neural network architectures (Multilayer Perceptron, Deep Convolutional Neural Network and Long-Short Term Memory), and found that the deep neural networks performed the best. 

Previous researchers compared the performance of traditional machine learning methods (k-NN, SVM) to a residual network (ResNet) \cite{ferrari2019hand}. They used a variety of datasets, including Motion Sense \cite{malekzadeh2018protecting}. They used only the accelerometer data, and compared the classifier performance for using the raw data vs hand-crafted features. For the motion sense dataset, the hand-crafted features gave the best results for the classifier, but the ResNet still performed the best. ResNets allow for skipping between layers; these skips contain non-linearities, which are better able to represent the non-linear accelerometer data \cite{he2016deep}.

While both papers proved that deep learning architectures are the superior methods for HAR, neither of their traditional approaches accounted for the non-linear aspect of the accelerometer data. We are interested in improving the performance of traditional machine learning methods by accurately accounting for the the non-linearity of the data in a pre-processing step. 

We used the the Motion Sense dataset, which contains time-series data from accelerometers and gyroscope sensors on an iPhone 6. The data was collected at a rate of 50 Hz. There are 24 participants of a varying age, gender, weight, and height. The participants performed 6 different activities: walking, jogging, sitting, standing, walking up stairs, and walking down stairs. 

The accelerometer and gyroscope data are both made of 3 components along the x, y, and z axes. The acceleration data is split based on gravity and user acceleration, each of which has 3 additional components along each axis. The gyroscope data reports roll, pitch, and yaw as well as rotation rate along the three axes.


\section{The Game Plan}
\label{sec:game_plan}
  I think this is the story we largely are trying to tell: 
\begin{enumerate}
	\item Can we use cell phone data to accurately and consistently distinguish between the activities people are doing? The answer had better be yes because it seems loads of other folks have done so.

		\begin{itemize}
	\item What techniques are used to do this? Luckily for us, we have this wonderful set of \emph{labeled} (and somewhat diverse) data. So obviously we are probably interested in looking at how supervised learning techniques perform. The amazing thing is that scikit-learn's syntax is basically identical once the data gets set up so it really becomes a plug and chug type of thing after that. Some techniques includes:
			\item Support vector machines (SVM)
			\item Some deep/crazy-architectured neural network; since this seems to be all the rage these days and since NN claim to be able to catch non-linearness of problems, it seems like a good thing to compare our results to.
			\item  Decision trees?
			\item SGD classifiers
			\item others?
		\end{itemize}

	\item But, what I think Prof.\ Gu is looking for is that we can use manifold learning with linear techniques for an unsupervised set of data. Can we classify accuately with that? Some techniques we may consider
		\begin{itemize}
			\item k-means
			\item PCA
			\item Gaussian mixture
		\end{itemize}

\end{enumerate}


\section{Our Data}
\label{sec:data}
  We used the MotionSense data. Blah blah blah.


\section{Citations, figures, tables, references}
\label{others}

These instructions apply to everyone.

\subsection{Citations within the text}

The \verb+natbib+ package will be loaded for you by default.
Citations may be author/year or numeric, as long as you maintain
internal consistency.  As to the format of the references themselves,
any style is acceptable as long as it is used consistently.

The documentation for \verb+natbib+ may be found at
\begin{center}
  \url{http://mirrors.ctan.org/macros/latex/contrib/natbib/natnotes.pdf}
\end{center}
Of note is the command \verb+\citet+, which produces citations
appropriate for use in inline text.  For example,
\begin{verbatim}
   \citet{hasselmo} investigated\dots
\end{verbatim}
produces
\begin{quote}
  Hasselmo, et al.\ (1995) investigated\dots
\end{quote}

If you wish to load the \verb+natbib+ package with options, you may
add the following before loading the \verb+nips_2016+ package:
\begin{verbatim}
   \PassOptionsToPackage{options}{natbib}
\end{verbatim}

If \verb+natbib+ clashes with another package you load, you can add
the optional argument \verb+nonatbib+ when loading the style file:
\begin{verbatim}
   \usepackage[nonatbib]{nips_2016}
\end{verbatim}

As submission is double blind, refer to your own published work in the
third person. That is, use ``In the previous work of Jones et
al.\ [4],'' not ``In our previous work [4].'' If you cite your other
papers that are not widely available (e.g., a journal paper under
review), use anonymous author names in the citation, e.g., an author
of the form ``A.\ Anonymous.''

\subsection{Footnotes}

Footnotes should be used sparingly.  If you do require a footnote,
indicate footnotes with a number\footnote{Sample of the first
  footnote.} in the text. Place the footnotes at the bottom of the
page on which they appear.  Precede the footnote with a horizontal
rule of 2~inches (12~picas).

Note that footnotes are properly typeset \emph{after} punctuation
marks.\footnote{As in this example.}

\subsection{Figures}

All artwork must be neat, clean, and legible. Lines should be dark
enough for purposes of reproduction. The figure number and caption
always appear after the figure. Place one line space before the figure
caption and one line space after the figure. The figure caption should
be lower case (except for first word and proper nouns); figures are
numbered consecutively.

You may use color figures.  However, it is best for the figure
captions and the paper body to be legible if the paper is printed in
either black/white or in color.
\begin{figure}[h]
  \centering
  \missingfigure{We would put a figure here of xyz}
  \caption{Sample figure caption.}
  \label{fig:example}
\end{figure}

\subsection{Tables}

All tables must be centered, neat, clean and legible.  The table
number and title always appear before the table.  See
Table~\ref{sample-table}.

Place one line space before the table title, one line space after the
table title, and one line space after the table. The table title must
be lower case (except for first word and proper nouns); tables are
numbered consecutively.

Note that publication-quality tables \emph{do not contain vertical
  rules.} We strongly suggest the use of the \verb+booktabs+ package,
which allows for typesetting high-quality, professional tables:
\begin{center}
  \url{https://www.ctan.org/pkg/booktabs}
\end{center}
This package was used to typeset Table~\ref{sample-table}.

\begin{table}[t]
  \caption{Sample table title}
  \label{sample-table}
  \centering
  \begin{tabular}{lll}
    \toprule
    \multicolumn{2}{c}{Part}                   \\
    \cmidrule{1-2}
    \textbf{Name}     & \textbf{Description}     & \textbf{Size} (\si{\micro\m}) \\
    \midrule
    Dendrite & Input terminal  & $\sim$100     \\
    Axon     & Output terminal & $\sim$10      \\
    Soma     & Cell body       & up to $10^6$  \\
    \bottomrule
  \end{tabular}
\end{table}


\subsubsection*{Acknowledgments}

We would like to thank people. Probably our lord and savior Prof.\ Gu for teaching us everything we know about nonlinear data analysis.\cite{dyson1970correlations}


%%%%%%%%%%%%%%%%%%%%%%%%
%%%% REFERENCES
%%%%%%%%%%%%%%%%%%%%%%%%

\bibliographystyle{unsrt}
\bibliography{biblio}


\end{document}
